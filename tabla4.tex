\documentclass[10pt,a4paper]{article}
\usepackage[utf8]{inputenc}
\usepackage[spanish]{babel}
\usepackage{amsmath}
\usepackage{amsfonts}
\usepackage{amssymb}
\usepackage{graphicx}
\usepackage{xcolor, colortbl}
\usepackage{lscape}
\usepackage{array, multirow, multicol}
\usepackage{geometry}

\definecolor{skyblue6}{rgb}{.2, .6, .8}
\definecolor{firebrick}{rgb}{.7, .13, .13}
\definecolor{blueice}{rgb}{.85, .96, .94}
\definecolor{lightcopper}{rgb}{.93, .76, .58}
\definecolor{blue-istat}{RGB}{0,44,89} 

\geometry{
 a4paper,
 total={170mm,257mm},
 left=27mm,
 right=25mm,
 top=25mm,
 bottom=20mm,
 }

\begin{document}
	\begin{table}[ht]
		\centering
		\begin{tabular}{m{2cm} m{4cm} m{3cm} m{3cm} m{3cm} }
			\hline
			\rowcolor{blue-istat}
			\multicolumn{1}{c} {\textcolor{white}{Aspecto}} & 
			\multicolumn{1}{c} {\textcolor{white}{Fortaleza}} & 
			\multicolumn{1}{c} {\textcolor{white}{Limitaciones}} & 
			\multicolumn{1}{c} {\textcolor{white}{Oportunidades}} & 
			\multicolumn{1}{c} {\textcolor{white}{Amenazas}} \\
			\hline
			Gestión institucional 
			&
			Institución educativa superior adventista. \par
			Formación técnica con valores cristianos.  \par
			Personal directivo y administrativo con experiencia en gestión educativa. \par
			Infraestructura adecuada y condiciones para una formación integral y el bienestar de sus estudiantes, docentes y personal administrativo. \par
			Servicios básicos disponibles. \par
			Personal comprometido con la institución. \par
			Cuenta con el respaldo de la UPeU.
			& 
			No estar posicionado en el mercado. \par
			Denominación institucional nueva. \par
			Poca experiencia del personal administrativo en la gestión operativa y procedimental de institutos.
			& 
			Demanda del mercado laboral y productivo de profesionales técnicos. \par
			Construir redes de apoyo a través de alianzas con entidades locales, regional y nacional. \par 
			Mayor posibilidad de inserción de profesionales técnicos.  \par 
			Ofrecer un servicio complementario a la formación distinta a la competencia. \par 
			& Existencia de la competencia en el entorno con propuesta educativa similar. \par 
			Cambios en la legislación nacional y normativa del sector educativo. \par 
			Sobre oferta en el mercado laboral de profesionales a fines a la propuesta educativa. \par 
			Cambio en las políticas económicas y disminución del poder adquisitivo de la población. \par \\
			\hline
			Gestión Académica
			& 
			Docentes con experiencia profesional en su especialidad. \par
			Cuenta con equipamiento esencial para la formación.  \par 
			Propuesta curricular educativa acorde a las exigencias actuales. \par 
			Respaldo de la propuesta curricular para continuidad de estudios con convalidación a la universidad. \par 
			Cuenta con servicios de acompañamiento y bienestar estudiantil. \par
			& 
			Carencia en la gestión operativa de los directivos de unidades académicas. \par
			Algunos docentes tienen el paradigma de educación universitaria con bajo enfoque a la formación práctica \par
			& 
			Formación y enseñanza acorde a la propuesta académica innovadora. \par 
			Desarrollo de estudiantes con énfasis para la empleabilidad y acciones sociales . \par
			& 
			Cambios tecnológicos y educativos vertiginosos  \par 
			Cambios en la legislación del sector que afecten a la propuesta educativa \par 
			Preferencia de cliente potencial por instituciones publicas  \par  \\
			\hline
					\end{tabular}
		\caption{Tabla simple}
	\end{table}
\end{document}